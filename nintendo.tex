\documentclass{article}

\title{Marktrelevantie van Nintendo sinds de oprichting}
\date{10 oktober 2017}
\author{Richard van Dijk}
\usepackage{hyperref}
\usepackage[margin=1.25in]{geometry}
\begin{document}
\maketitle






\section{Inleiding}

Nintendo is een oud bedrijf dat aan de wieg van de moderne cultuur rondom videospellen heeft gestaan. In al die jaren is er veel veranderd. Computers zijn sterker en sneller geworden. Ook is de markt en dus de doelgroep van spellen veel breder geworden. Met zoveel veranderingen op de markt betekent stilstand achteruitgang. In dit verslag wordt uiteengezet hoe Nintendo zich door de jaren heeft aangepast aan de veranderende markt. Het doel is een chronologische weergave te bieden van de vernieuwingen die Nintendo door de jaren heeft doorgevoerd om actueel te blijven. Daarbij wordt als hoofdvraag gesteld:
\begin{itemize} \item Hoe heeft Nintendo sinds de oprichting marktrelevantie behouden? \end{itemize}
Deze vraag zal verder opgesplitst worden in vier deelvragen:
\begin{itemize}
\item Hoe is Nintendo in de video game markt gekomen?
\item Welke ontwerpvernieuwingen heeft Nintendo gedaan?
\item Welke gameplayvernieuwingen heeft Nintendo gedaan?
\item Hoe is de marketingstrategie van Nintendo door de jaren heen veranderd?
\end{itemize}

\section{Oprichting en eerste werkzaamheden}
Nintendo is in 1889 opgericht door Fusajiro Yamauchi. In het begin hield het bedrijf zich bezig met de vervaardiging van handgemaakte hanafuda-speelkaarten. Enkele jaren later, in 1902 gingen ze ook aan de slag met het maken van westerse speelkaarten. Over de volgende 50 jaren werden er verschillende werkplaatsen geopend en werd er o.a. een distributiebedrijf geopend. In 1953 begon Nintendo met het maken van plastic speelkaarten, en in 1959 boorden ze de kindermarkt aan met kaarten met Disneyplaatjes erop. 

\section{Switch van kaarten naar andere bezigheden}
In 1963 begon Nintendo met het experimenteren met andere bezigheden naast hun bestaande business, omdat het bedrijf besefte dat er in de kaartenmarkt beperkte groei mogelijk was. Verschillende richtingen werden geprobeerd. Zo werd er onder andere een taxibedrijf opgericht, een zogenaamd love hotel geopend, en verkochten ze snelkookrijst. Deze bedrijven waren in meer of mindere mate succesvol, maar bleken uiteindelijk toch niet permanent te zijn. 
In deze periode maakte Nintendo ook hun eerste speelgoed. Na een paar eerste producten bracht het bedrijf in 1966 de succesvolle Ultra-Hand uit, een grijphand die via een lattensysteem uitrekte. Het speelgoed was ontworpen door Gunpei Yokoi, die na dit succes meer speelgoed ontwierp en op de markt bracht.


\section{Elektronica en vroege videogames}
Rond 1970 waagde Nintendo zich voor het eerst aan elektronische apparaten. Het bedrijf ontwierp de Beam Gun-spellen, die lichtgevoelige elektronica bevatten. Op dit concept voortbordurend introduceerde Nintendo in 1973 een laserschietbaan, waar men op bewegende doelen kon schieten. Later werd een kleinere versie ontwikkeld die op arcades ingezet kon worden.
In dezelfde jaren kocht Nintendo de distributierechten voor de Magnavox Odyssey, de eerste videogameconsole die commercieel verkocht werd. Twee jaar later ontwikkelde Nintendo samen met Mitsubishi Electric de eerste eigen consoles, de Color TV-Game 6 en Color TV-Game 15, die een week van elkaar uitgebracht werden. Beide consoles bevatten respectievelijk 6 en 15 variaties van het spel Pong. In 1978 werd de Color TV-Game 112 uitgebracht; dit was een racespel en was het eerste project van ontwikkelaar Shigeru Miyamoto. Nog twee versies werden uitgebracht die meer of minder succesvol waren. 
\\ 
Tegelijk met de homeconsoles maakte Nintendo ook arcademachines. Onder andere het iconische spel Donkey Kong zag in deze periode het licht. Opnieuw was dit ontworpen door Miyamoto, en het bleek een grote hit te zijn in Amerika en Japan.
\\ 
Aan het eind van de jaren '70 bracht Nintendo ook de eerste exemplaren van de Game\&Watch serie op de markt. Dit waren de eerste draagbare elektronische spelletjes, bedacht door Gunpei Yokoi. Tot de jaren '90 zou een grote reeks van deze spellen geproduceerd worden.

\section{Video game crash}
In 1983 crashte de markt voor gameconsoles in Amerika. Door een combinatie van factoren (oververzadiging van de markt beide in console-aanbod en spelaanbod, de opkomst van de home computer, inflatie) zakte het vertrouwen in gameconsoles als een plumppudding in elkaar. Veel ontwikkelaars gingen failliet en grote hoeveelheden onverkoopbare games werden gedumpt. In Japan had Nintendo ondertussen een console met de naam Family Computer (afgekort als Famicom) uitgebracht die na een langzame start erg populair bleek. Nadate een pogin om samen met Atari een microcomputer gebaseerd op de Famicom in Amerika uit te brengen op de valreep strandde, besloot Nintendo om de computer zelf op de markt te brengen. Het publiek reageerde in Amerika gematigd en Nintendo trok zich terug, om later op de markt te komen met een eenvoudigere versie: de NES. 
Om een herhaling van de 

\section{Consoles}
Vanaf de jaren '80 heeft Nintendo een groot aantal al dan niet draagbare spelcomputers gemaakt. Hieronder worden de belangrijkste consoles kort besproken. 
\subsection{Home consoles}


\subsubsection{Nintendo Entertainment System / Family Computer}
De eerste 
\subsubsection{Super NES / Super Famicom}
De Super NES
\subsubsection{Nintendo 64}
n64
\subsubsection{GameCube}
gc
\subsubsection{Wii}
wii
\subsubsection{Wii U}
wii u
\subsubsection{Switch}
switch

\subsection{Portables}
placeholder
\subsubsection{Game \& Watch}
g\&w
\subsubsection{Game Boy}
gb
\subsubsection{Virtual Boy}
vb
\subsubsection{Game Boy Color}
gbc
\subsubsection{Game Boy Advance}
gba
\subsubsection{Nintendo DS}
nds
\subsubsection{Nintendo 3DS}
n3ds
\section{Conclusie}
KANKER
\section{Bronvermelding}
\url{https://www.nintendo.co.uk/Corporate/Nintendo-History/Nintendo-History-625945.html}\\ 
\url{http://blog.beforemario.com/2012/09/nintendo-kousenjuu-duck-hunt-1976.html}\\ 
\url{https://books.google.nl/books?id=b_N5FzzD3hsC&pg=PT40}\\ 

\end{document}


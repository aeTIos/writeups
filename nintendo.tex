\documentclass{article}

\title{Marktrelevantie van Nintendo sinds de oprichting}
\date{10 oktober 2017}
\author{Richard van Dijk}
\usepackage{hyperref}
\usepackage[margin=1.25in]{geometry}
\renewcommand{\contentsname}{Inhoud}
\pagenumbering{gobble}
\begin{document}
\maketitle
\newpage
\tableofcontents
\newpage



\pagenumbering{arabic}
\setcounter{page}{1}
\section{Inleiding}

Nintendo is een oud bedrijf dat aan de wieg van de moderne cultuur rondom videospellen heeft gestaan. In al die jaren is er veel veranderd. Computers zijn sterker en sneller geworden. Ook is de markt en dus de doelgroep van spellen veel breder geworden. Met zoveel veranderingen op de markt betekent stilstand achteruitgang. In dit verslag wordt uiteengezet hoe Nintendo zich door de jaren heeft aangepast aan de veranderende markt. Het doel is een chronologische weergave te bieden van de vernieuwingen die Nintendo door de jaren heeft doorgevoerd om actueel te blijven. Daarbij wordt als hoofdvraag gesteld:
\begin{itemize} \item Hoe heeft Nintendo sinds de oprichting marktrelevantie behouden? \end{itemize}
Deze vraag zal verder opgesplitst worden in vier deelvragen:
\begin{itemize}
\item Hoe is Nintendo in de video game markt gekomen?
\item Welke ontwerpvernieuwingen heeft Nintendo gedaan?
\item Welke gameplayvernieuwingen heeft Nintendo gedaan?
\item Hoe is de marketingstrategie van Nintendo door de jaren heen veranderd?
\end{itemize}

\section{Oprichting tot 1983}
\subsection{Oprichting en eerste werkzaamheden}
Nintendo werd in 1889 opgericht door Fusajiro Yamauchi. In het begin hield het bedrijf zich bezig met de vervaardiging van handgemaakte hanafuda-speelkaarten. Enkele jaren later, in 1902 gingen ze ook aan de slag met het maken van westerse speelkaarten. Over de volgende 50 jaren werden er verschillende werkplaatsen geopend en werd er o.a. een distributiebedrijf geopend. In 1953 begon Nintendo met het maken van plastic speelkaarten, en in 1959 boorden ze de kindermarkt aan met kaarten met Disneyplaatjes erop. 

\subsection{Switch van kaarten naar andere bezigheden}
In 1963 begon Nintendo met het experimenteren met andere bezigheden naast hun bestaande business, omdat het bedrijf besefte dat er in de kaartenmarkt beperkte groei mogelijk was. Verschillende richtingen werden geprobeerd. Zo werd er onder andere een taxibedrijf opgericht, een zogenaamd love hotel geopend, en verkochten ze snelkookrijst. Deze bedrijven waren in meer of mindere mate succesvol, maar bleken uiteindelijk toch niet permanent te zijn. 
In deze periode maakte Nintendo ook hun eerste speelgoed. Na een paar eerste producten bracht het bedrijf in 1966 de succesvolle Ultra-Hand uit, een grijphand die via een lattensysteem uitrekte. Het speelgoed was ontworpen door Gunpei Yokoi, die na dit succes meer speelgoed ontwierp en op de markt bracht.


\subsection{Elektronica en vroege videogames}
Rond 1970 waagde Nintendo zich voor het eerst aan elektronische apparaten. Het bedrijf ontwierp de Beam Gun-spellen, die lichtgevoelige elektronica bevatten. Op dit concept voortbordurend introduceerde Nintendo in 1973 een laserschietbaan, waar men op bewegende doelen kon schieten. Later werd een kleinere versie ontwikkeld die op arcades ingezet kon worden.

In dezelfde jaren kocht Nintendo de distributierechten voor de Magnavox Odyssey, de eerste videogameconsole die commercieel verkocht werd. Twee jaar later ontwikkelde Nintendo samen met Mitsubishi Electric de eerste eigen consoles, de Color TV-Game 6 en Color TV-Game 15, die een week van elkaar uitgebracht werden. Beide consoles bevatten respectievelijk 6 en 15 variaties van het spel Pong. In 1978 werd de Color TV-Game 112 uitgebracht; dit was een racespel en was het eerste project waar ontwikkelaar Shigeru Miyamoto aan meewerkte. Nog twee versies werden uitgebracht die meer of minder succesvol waren. 
\\ 
Tegelijk met de homeconsoles maakte Nintendo ook arcademachines. Onder andere het iconische spel Donkey Kong zag in deze periode het licht. Opnieuw was dit ontworpen door Miyamoto, en het bleek een grote hit te zijn in Amerika en Japan.
\\ 
Aan het eind van de jaren '70 bracht Nintendo ook de eerste exemplaren van de Game\&Watch serie op de markt. Dit waren de eerste draagbare elektronische spelletjes, bedacht door Gunpei Yokoi. Tot de jaren '90 zou een grote reeks van deze spellen geproduceerd worden.

\subsection{Video game crash}
In 1983 crashte de markt voor gameconsoles in Amerika. Door een combinatie van factoren (overver-zadiging van de markt beide in console-aanbod en spelaanbod, de opkomst van de home computer, inflatie) zakte het vertrouwen in gameconsoles in elkaar.
Veel ontwikkelaars gingen failliet en grote hoeveelheden onverkoopbare games werden gedumpt. In Japan had Nintendo ondertussen een console met de naam Family Computer (afgekort als Famicom) uitgebracht die na een langzame start erg populair bleek. Nadat een poging om samen met Atari een microcomputer gebaseerd op de Famicom in Amerika uit te brengen op de valreep strandde, besloot Nintendo om de computer zelf op de markt te brengen. Het publiek reageerde in Amerika gematigd en Nintendo trok zich terug, om later op de markt te komen met een eenvoudigere versie: de NES. 

\section{Consoles}
Vanaf de jaren '80 heeft Nintendo een groot aantal al dan niet draagbare spelcomputers gemaakt. Hieronder worden de belangrijkste consoles en hun eigenschappen kort toegelicht. 
\subsection{Home consoles}
\subsubsection{1983: Nintendo Entertainment System / Family Computer}
De Famicom was een revolutie voor Nintendo. Het was hun eerste console die gebruik maakte van verwisselbare cartridges. Daarnaast kon de console in vergelijking met de voorheen populaire generatie een groot aantal kleuren en plaatjes op het scherm weergeven. Deze beide vernieuwingen zorgden voor een veel groter aanbod in spellen.
Een vernieuwing in de gamewereld was de controller. Deze maakte niet zoals veel andere consoles gebruik van een joystick, maar van een zogenaamd D-pad. Het ontwerp hiervan (een kruisvormige knop) was geleend van de Game\&Watch spelletjes, en bleek een succes. \\ 
De introductie van de Amerikaanse versie van de console was een uitdaging door het afgenomen vertrouwen in de videogameindustrie in het land. De versie die Nintendo uiteindelijk op de markt bracht in Amerika was zo ontworpen dat het zo min mogelijk leek op een `traditionele' gameconsole. Ook werden in de marketing termen vermeden die geassocieerd werden met consoles. Zo werden de cartridges `Paks' genoemd, de console zelf een `Control Deck'. Tevens werd de NES meer als speelgoed in de markt gezet, met accessoires als de Zapper en de robot R.O.B. \\ 
Nintendo, dat geleerd had van de instorting van de bubbel enkele jaren eerder, had ook een manier gevonden om te voorkomen dat de markt opnieuw verzadigd zou raken. In de NES zat een zogenaamde `lockout chip', waardoor alleen ontwikkelaars die van Nintendo een licentie hadden gekregen games op het systeem uit konden brengen. Tevens bevatte de licentieovereenkomst clausules die moesten voorkomen dat ontwikkelaars teveel spellen op de markt brachten, of spellen op andere consoles uitbrachten. Nintendo prees de resulterende hogere kwaliteit van spellen aan met het Nintendo Seal of Quality, dat op de doos en de voorkant van cartridges stond.\\ 
De NES liep uit op een groot succes, met een groot aantal iconische spellen die op het systeem hun levenslicht zagen. Mario, Zelda en Metroid zijn voorbeelden van franchises die hun stempel op de industrie gedrukt hebben.

\subsubsection{1990: Super NES / Super Famicom}
Eind jaren `80 werd Nintendo door de concurrentie voor de NES door nieuwere, meer capabele consoles genoodzaakt een nieuw model te ontwikkelen om marktaandeel te houden. Dit resulteerde in de Super Famicom, later in de VS en EU uitgebracht als de Super NES. 
De nieuwe console bracht, naast natuurlijk een sterkere processor, enkele nieuwe dingen mee. De controller had een voor die tijd uniek ontwerp met meer knoppen op de voorzijde en twee schouderknoppen, een ontwerp dat sindsdien door vrijwel elke controller geïmiteerd is.
Ook konden SNES-spellen in de cartridge extra processors ingebouwd hebben voor extra functionaliteit. Zo kon bijvoorbeeld het spel Star Fox als één van de eerste consolegames gebruikmaken van 3d-graphics.


\subsubsection{1996: Nintendo 64}
3D-graphics waren op de Nintendo 64 de focus. De Nintendo 64 viel op in de markt omdat het de eerste grote console was die een analoge stick had. Andere eigenschappen van de console waren de mogelijkheid om in de controller uitbreidingen te stoppen, zoals de Rumble Pak. Ook was de Nintendo 64 uniek omdat de console niet de gebruikelijke 2, maar 4 controllerpoorten had, zodat groepen mensen zonder adapter samen konden spelen. De nintendo 64 was het platform voor twee van de meest gevierde spellen ooit: Super Mario 64 en The Legend of Zelda: Ocarina of Time. 

\subsubsection{GameCube}
Met de GameCube bracht Nintendo een sterke console op de markt
analoge triggers
draagbaar


\subsubsection{Wii}
wii
motion controls
marketing richting families en niet gamers
accessoires
gebruikt vrijwel dezelfde hw als gc

\subsubsection{Wii U}
wii u
hd graphics

\subsubsection{Switch}
switcht tussen portable en console
botw launch boost 
\subsection{Portables}
placeholder
\subsubsection{Game \& Watch}
g\&w

Donkey Kong
\subsubsection{Game Boy}
gb
gebaseerd op cartridgesysteem van nes
tetris
connectiemogelijkheden (IR, kabel)
\subsubsection{Virtual Boy}
vb
3d
flop
\subsubsection{Game Boy Color}
gbc
\subsubsection{Game Boy Advance}
gba
gbasp
\subsubsection{Nintendo DS}
nds
2 schermen
touch screen
casual gamer markt 
ndslite
\subsubsection{Nintendo 3DS}
o3ds
n3ds
3d scherm
meer hardcore games

\section{Conclusie}
Conclusie
\section{Bronvermelding}
\url{https://www.nintendo.co.uk/Corporate/Nintendo-History/Nintendo-History-625945.html}\\ 
\url{http://blog.beforemario.com/2012/09/nintendo-kousenjuu-duck-hunt-1976.html}\\ 
\url{https://books.google.nl/books?id=b_N5FzzD3hsC&pg=PT40}\\ 
\url{https://en.wikipedia.org/wiki/List_of_video_games_considered_the_best#Top_ranks}

\end{document}

